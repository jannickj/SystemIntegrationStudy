\begin{description}
\item [{Toll~tag}] A toll tag is a physical device which will admit passage
through an express lane when equipped on a vehicle. Toll tags need
to be ordered by customers, and can only be used by one vehicle. The
toll system automatically computes the fare of the vehicle based on
the distance traveled between check in and check out.

\item [{Toll~Station}] A toll station is a collection of toll lanes. A
station manager is connected to every toll station and can view \emph{statistical
reports}, as well as view and modify customer data.

\item [{Statistical~Report}] Each toll station records which vehicles
purchased single tickets and used toll tags, along with the type of
the vehicle. A statistical report is a write-out of this information
for a given time period.

\item [{Express~Lane}] An express lane will automatically admit passage
to vehicles equipped with a toll tag. The vehicle will be registered
as having passed through the lane at the given time.

\item [{Credit~Card~Lane}] In a credit card lane, the user can use their
credit card to purchase a single ticket.

\item [{Cash~Lane}] A cash lane is manned by a cashier, who accepts cash
payments as well as credit cards. When a cashier mans a credit card
lane, it becomes a cash lane.

\item [{Normal Lane}] A normal lane is an umbrella term for non-express lanes.

\item [{Cashier}] A cashier operates a cash lane and accepts cash payment
in exchange for single tickets.

\item [{Customer}] The customer is the driver of a vehicle.

\item [{Rate}] The price to pay for a certain vehicle type for certain ticket/tag type.

\item [{Enterprise}] The company which control these sections of the motorway.

\item [{Enterprise manager}] The manager who is in charge of all the toll stations.

\end{description}
