\begin{figure}[H]
\centerline{\includegraphics[width=1.4\columnwidth]{"img/sequence_diagrams/SequenceDiagram_CheckInSingleTicket_cash"}}
\caption{Sequence diagram for main scenario of use case "Check-In Single Ticket"}
\label{fig:seq_single_ticket_main}
\end{figure}

\begin{enumerate}
\item The customer is at a cash lane.
\item The cashier selects the customer's vehicle type, \texttt{vt}, on the touch screen using \texttt{selectVehicleType(vt)}.
	\begin{enumerate}
	\item This results in the touch screen calling \texttt{getVehiclePrice(vt)} on \texttt{TollLaneNormalCheckIn}, ie. the toll lane computer. 
	\item The \texttt{price} is returned and is presented to the cashier by being shown on the touch screen, 
	\item The toll lane computer awaits a payment.
	\end{enumerate}
\item The cashier recieves appropriate cash payment from the customer, and deposits it in the cash register, (calling \texttt{depositCash()})
	\begin{enumerate}
	\item The cash register calls the \texttt{cashReceived(price)} method on the toll lane computer.
	\item \label{generate_ticket} The toll lane computer generates a ticket (\texttt{ticket}) using the \texttt{generateTicket(price)} method, which constructs a ticket with the current time stamp and a unique ID based on the toll lane's ID in the system.
	\item The station server is informed of the sale when the toll lane computer calls \texttt{ss.ticketSold(ticket)}.
		\begin{enumerate}
		\item The station server stores the ticket in the ticket database.
		\item The station server informs the enterprise server of the sale by calling \texttt{es.ticketSold(ticket)}.
		\end{enumerate}
	\item The generated ticket is sent to the ticket printer using the \texttt{printer.printTicket(ticket)} method. The ticket printer prints the ticket.
	\item The toll lane computer calls \texttt{barrier.open()}, causing the barrier to open.
	\item The toll lane computer calls \texttt{delay()} delaying execution for a fixed time, allowing the vehicle to pass through.
	\end{enumerate}
	\item The customer leaves the lane.
	\begin{enumerate}
	\item After the call to \texttt{delay()} returns, the toll lane computer calls \texttt{barrier.close()}, and the barrier closes.
	\end{enumerate}
	
\end{enumerate}


\subsubsection{Check-In Single Ticket Alternative Scenario}

\begin{figure}
\centerline{\includegraphics[width=1.4\columnwidth]{"img/sequence_diagrams/SequenceDiagram_CheckInSingleTicket_card"}}
\caption{Sequence diagram for alternative scenario of use case "Check-In Single Ticket"}
\label{fig:seq_single_ticket_alt}
\end{figure}

\begin{enumerate}
\item The customer approaches the lane.
\item The customer selects their vehicle type, \texttt{vt}, on the touch screen using \texttt{selectVehicleType(vt)}.
	\begin{enumerate}
	\item The touch screen calls \texttt{getVehiclePrice(vt)} on the toll lane computer, which returns the price for the vehicle.
	\item The touch screen shows the returned price.
	\end{enumerate}
\item The customer presents his credit card \texttt{cardInfo} to the credit card reader (via \texttt{swipeCard(cardInfo)}).
	\item The credit card reader calls the method \texttt{cardInfoReceived(cardInfo)} provided by the toll lane computer.
		\item The toll lane computer calls \texttt{bank.withdraw(cardInfo, price)}, which returns \texttt{true} as the money have successfully been transferred.
		\item The execution proceeds as in the use case validation for the main scenario from \autoref{generate_ticket}.
		
	
\end{enumerate}